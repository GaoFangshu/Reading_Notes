%-*- coding: UTF-8 -*-
\documentclass{article}

\usepackage{indentfirst}
\setlength{\parindent}{0.5cm}

\usepackage{graphicx}
\usepackage[top=1in, bottom=1.25in, left=1.25in, right=1.25in]{geometry}

%\usepackage{setspace}
%\renewcommand{\baselinestretch}{1.3}

%\usepackage{titling}
%\setlength{\droptitle}{-2cm}

\usepackage{subcaption} 
\usepackage{float}

\title{Reading Notes of \textit{Actual Causality}}
\author{Gao Fangshu}
\date{2019.4.16}

\begin{document}
\maketitle
\section*{Chapter 1. Introduction and Overview}
There are two notions of caulsality:
\begin{itemize}
\item \textit{type causality}: Also called \textit{general causality}. Type causality contains general staments, and allows people to make predictions (\textit{forward-looking}). 
\item \textit{actual causality}: Also called \textit{token causality} or \textit{specific causality}. Actual causality focus on particular events, and related to words such as "responsebility" and "blame".
\end{itemize}

Roughly speaking, reasoning about type causality is equivalent to reasoning about \textit{effects of causes} (possible effects of a given event), whereas reasoning about actual causality is equivalent to reasoning about \textit{causes of effects} (possible causes of a particular outcome).

``But-for" definition of causality: A is a cause of B if, but for A, B would not have happened. However, it is not always enough to determine causality, and \textit{Halpern-Pearl definition} solve some problems where but-for test fails.

\section*{Chapter 2. The HP Definition of Causality}
Definition: A \textit{causal model} $M$ is a pair $(\mathcal{S}, \mathcal{F})$:
\begin{itemize}
	\item $\mathcal{S}$: A \textit{signature}, explicitly lists the endogenous and exogenous variables and characterizes their possible values. A signature $\mathcal{S}$ is a tuple ($\mathcal{U}$,$\mathcal{V}$,$\mathcal{R}$):
	\begin{itemize}
		\item $\mathcal{U}$: A set of exogenous variables
		\item $\mathcal{V}$: A set of endogenous variables
		\item $\mathcal{R}$: $\mathcal{R}$ maps variables in $\mathcal{U}$ or $\mathcal{V}$ into possible values for them (i.e., the set of values over which the variable ranges).
	\end{itemize}
	\item $\mathcal{F}$: A set of \textit{structural equations}. $\mathcal{F}$ associates with each endogenous variable $X \in \mathcal{V}$ a function denoted $F_{X}$ maps $\times_{Z \in(\mathcal{U} \cup \mathcal{V}-\{X\})} \mathcal{R}(Z)$ to $\mathcal{R}(X)$. That means, function $F_{X}$ captures a relation between all varaibles but for $X$ and $X$. $F_{X}$ determines the value of $X$, given the values of all the other variables in $\mathcal{U} \cup \mathcal{V}$. 
\end{itemize}



\end{document}